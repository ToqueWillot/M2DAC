\documentclass[a4paper,11pt]{report}
\usepackage[utf8]{inputenc}
\usepackage[francais]{babel}
\usepackage{graphicx}
\usepackage{epsf}
\usepackage{epsfig}
\usepackage{listings}
\usepackage{color}
\usepackage{graphicx}
\usepackage{wrapfig}
\usepackage{caption}
\usepackage{subcaption}
\definecolor{dkgreen}{rgb}{0,0.6,0}
\definecolor{gray}{rgb}{0.5,0.5,0.5}
\definecolor{mauve}{rgb}{0.58,0,0.82}
\usepackage{amssymb}
\setcounter{tocdepth}{5}
\setcounter{secnumdepth}{5}
\providecommand{\keywords}[1]{\textbf{\textit{Index terms---}} #1}
\usepackage{geometry}
\geometry{ hmargin=2.5cm, vmargin=2.5cm } 
\setcounter{tocdepth}{5}
\lstset{frame=tblr,
  language=Python,
  aboveskip=3mm,
  belowskip=3mm,
  showstringspaces=false,
  columns=flexible,
  basicstyle={\small\ttfamily},
  numbers=left,
  numberstyle=\tiny\color{gray},
  keywordstyle=\color{blue},
  commentstyle=\color{dkgreen},
  stringstyle=\color{mauve},
  breaklines=true,
  breakatwhitespace=true,
  tabsize=3
}
\newcommand{\bigO}[1]{\ensuremath{\mathop{}\mathopen{}O\mathopen{}\left(#1\right)}}
\pagestyle{plain}



\begin{document}

~\\\\\\\\\\
\begin{center}
\mbox{\huge RAPPORT de l'UE : RI - Recherche d'informations}\\~\\ \mbox{ }\\ \textbf{\LARGE Création d'un moteur de recherche d'images}
\end{center}~\\
\begin{center}\large Paul Willot et Florian Toqué\end{center}
\begin{center}\large 9 Novembre 2015 \end{center}~\\\\\\\\\\\\\\\\\
%\rule{\linewidth}{.1pt}
%\begin{center}Abstract\end{center}
\begin{center}\textbf{\Large{Résumé}}\end{center}
\large{Ce rapport aborde les différentes étapes que nous avons suivies pour construire les briques d'un moteur de recherche d'images.}\\\\\\\\\\\\\\\\\\\\\\\\
\rule{\linewidth}{.5pt}
\textsc{Mots-clés:}\\
Moteur de recherche d'image --- BOW --- SIFTS --- Apprentissage structuré --- Classifieur multi-classes et hiérarchique\\\\\\



\newpage
\tableofcontents







\chapter*{{\centering Introduction}}
\addcontentsline{toc}{chapter}{Introduction}
Les moteurs de recherche servent à retrouver de l'information parmi un ensemble de documents. Pour chaque requête posée, leur action est de retourner les documents les plus pertinents. 

\section*{1.Les images}
\addcontentsline{toc}{section}{1.Les images}

\subsection*{1.Collections de documents}
\addcontentsline{toc}{subsection}{1.Collections de documents}

\subsection*{2.Indexation visuelle}
\addcontentsline{toc}{subsection}{2.Indexation visuelle}

\section*{2.Algorithme d'apprentissage structuré}
\addcontentsline{toc}{section}{2.Algorithme d'apprentissage structuré}




\section*{3.Classifieur multi-classes}
\addcontentsline{toc}{section}{3.Classifieur multi-classes}

\subsection*{1.Evaluation du classifieur multi-classes}
\addcontentsline{toc}{subsection}{1.Evaluation du classifieur multi-classes}

\section*{4.Classifieur hiérarchique}
\addcontentsline{toc}{section}{4.Classifieur multi-classes}

\subsection*{1.Evaluation du classifieur hiérarchique}
\addcontentsline{toc}{subsection}{1.Evaluation du classifieur hiérarchique}




\section*{5.Modèles de ranking}
\addcontentsline{toc}{section}{5.Modèles de ranking}

\subsection*{1.Evaluation du ranking}
\addcontentsline{toc}{subsection}{1.Evaluation du ranking}

\chapter*{Conclusion}
\addcontentsline{toc}{chapter}{Conclusion}



%\section*{1.Nom\_section}
%\addcontentsline{toc}{section}{1.Nom\_section}

%\subsection*{1.Nom\_subsection}
%\addcontentsline{toc}{subsection}{1.Nom\_subsection}

%\subsubsection*{1.Nom\_subsubsection}
%\addcontentsline{toc}{subsubsection}{1.Nom\_subsubsection}





%ajouter des images png \hfill si ajout d'une autre photo à côté
%\includegraphics[width=0.5\textwidth]{nom_image}\hfill
%\includegraphics[width=0.5\textwidth]{nom_image}



\end{document}
