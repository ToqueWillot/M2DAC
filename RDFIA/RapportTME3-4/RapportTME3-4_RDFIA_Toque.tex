\documentclass[a4paper]{article}

\usepackage[english,francais]{babel}
\usepackage[utf8]{inputenc}
\usepackage{amsmath}
\usepackage{graphicx}
\usepackage[colorinlistoftodos]{todonotes}
\usepackage{float}


\title{Calcul de représentation d'images et apprentissage supervisé}

\author{Florian Toqué}

\date{20 Octobre, 2015}

\begin{document}

\maketitle

\selectlanguage{francais}
\begin{abstract}


Ce rapport porte sur un travail effectué au cours de l'UE RDFIA du Master informatique spécialité Imagerie de l'université Pierre et Marie Curie (Paris). Le travail réalisé consiste à mettre en place un système de classification d’images par apprentissage supervisé. Une fois appris ce système de classification est en mesure d’affecter une des 15 catégories sémantiques à une image de test donnée. Dans ce rapport nous expliquons une des parties nécessaire à ce système qui est la représentation d'image en Bag of Words et l'apprentissage supervisé (SVM). 
\end{abstract}

\section{Introduction}
Lors de ce projet nous avons implémenté deux parties, le calcul des BoW qui représentent les images et la formation d'un classifieur multi-classe (SVM en un contre tous) .


\section{Calcul de représentation d'image: le modèle Bag of Words (BoW)}
Afin de classifier des images il est tout d'abord nécessaire de les représenter de manière uniforme. La technique employée ici est de représenter chaque image sous forme de Bag of Words. Chaque image est tout d'abord représentée par un ensemble de SIFTS(contenant 128 mots visuels) calculés au préalable, nous avons aussi calculé à l'aide d'un algorithme de clustering, un dictionnaire contenant les mots les plus représentatifs de l'ensemble des images que nous avons à notre disposition. Le principe est d'assigner un poids à chaque mots de ce dictionnaire pour que le vecteur de mots visuels représente au mieux une image. 
\subsection{Visualisation du dictionnaire visuel}
Le dictionnaire visuel contient l'ensemble des "mots visuels" les plus représentatifs de l'ensemble de la base 15-scenes, calculé dans un travail précédent à l'aide de l'algorithme du KMeans. Voici sa visualisation:\todo{mettre limage du dictionnaire visuel} ici

\subsection{Calcul de BoW sur des images et visualisation des résultats}
Nous voulons représenter les images d'une manière uniforme pour pouvoir les classifier.\\
Pour ce faire deux étapes sont nécessaires, la première est le cooding, cette étape consiste à calculé pour chaque SIFT de l'image le mot visuel le plus similaire à celui dans le dictionnaire. Une fois cette étape effectuée il suffit de faire l'aggrégation (pooling) des mots trouvés pour chaque SIFTS et de diviser l'ensemble par le nombre de SIFTS de l'images. Nous obtenons alors la représentation d'une image sous forme de vecteur de la taille du dictionnaire contenant à l'emplacement de chaque mot visuel son importance pour la représentation de l'image sous forme de scalaire.\\
Voici la visualisation d'une image et de ses mots les plus fréquents ainsi que leurs localisations:\todo{mettre image testBowImage}ici

\subsection{Calcul de l’ensemble des BoW sur la base}
Afin de classifier les 4485 images de la bases nous devons calculer leurs BoW. Pour ce faire nous faisons comme précedemment pour toutes les images de la base en stockant leurs BoW dans un fichier.



\section{Apprentissage supervisé pour la classification sémantique d'images}
Chaque image est représentée sous forme d'un BoW qui est un vecteur de taille 1000. Pour classifier chaque image parmis les 15 catégories possible nous avons utiliser l'algorithme d'apprentissage SVM. Nous allons tout dabord présenter cet algorithme de classification pour des données binaires, puis nous expliquerons comment il est utilisé pour classifier des données multi-classes.
\subsection{Apprentissage des SVM binaires}
Un classifieur SVM(support vector machines) est un algorithme d(apprentissage linéaire. Il permet de créer une frontière entre deux classes différentes tout comme un algorithme de regression logistique. L'avantage est qu'en plus de séparer l'espace binaire en deux espaces distincts, il maximise la marge qui existe entre l'hyperplan séparateur est les données ce qui revient en fait à résoudre un problème d'optimisation sous contrainte.

Voici la visualisation de la séparatrice d'un svm pour des données binaires:\todo{mettre image svm avec marge maximisée}ici
\subsection{Formation d'un classifieur multi-classe}
Notre problème de classification d'image n'est pas une classification binaire, en effet nous devons catégoriser nos images parmi 15 classes différentes et non deux. Pour ce faire nous allons créer 15 classifieurs binaires chacun correspondant à une classe versus toutes les autres. Pour classifier une image nous classifierons l'image avec les 15 classifieurs et choisirons la classe dont le score sera maximal.

Voici la visualisation de la matrice de confusion obtenu après la catégorisation de notre ensemble de train:\todo{mettre image matrice de confusion}ici

Voici la visualisation de la matrice de confusion obtenu après la catégorisation de notre ensemble de test:\todo{mettre image matrice de confusion}ici
\subsection{Carte de chaleur de classification}
Afin d'analyser l'impact de chaque descripteur local lors de la classification d'image, nous avons mis en place une fonction permettant de calculer une "carte de chaleur de classification". Pour ce faire nous choisissons les descripteurs locaux ayant les plus fortes contributions. Nous avons donc calculé pour chaque descripteur local sa contribution lors de la classification à l'aide de la formule suivante:  $\frac{1}{N}  w_m^*$ \\\\
\textbf{Question: Montrer que la contribution de chaque descripteur $x_j$ dans la fonction de classification s'écrit $p(x_j) = \frac{1}{N}  w_m^*$} ?\\\\
Ceci est plus une explication qu'une démonstration..\\
N correspond au nombre de descripteurs locaux dans l'image et $m^*$ au mot visuel le plus proche de la classe k en question. Il suffit alors de regarder quel est le poids associé à ce mot visuel dans le modèle de classification de la classe k pour connaitre sa contribution. Etant donné qu'il s'agit d'une probabilité il faut divisé ce poids par le nombre de descripteurs locaux. Nous obtenons ainsi pour chaque descripteur local le mot visuel le plus discriminant pour la catégorisation avec une certaine probabilité. Ceci nous permet de faire cette visualisation qui représente pour chaque descripteur l'importance qu'il a dans la catégorisation. \\
\todo{mettre image de la carte de chaleur}



\section{Bonus: analyse de résultats et limitations de la représentation utilisée}
\subsection{Prise en compte de l'information spatiale dans la représentation BoW}
Dans le travail présenté jusqu'ici nous avons utilisé les BoW sans information spatiale, c'est à dire que la même image dupliquée d'une manière identique Image1 et dupliquée avec une rotation de manière inversée Image2 sera considérée comme la même image alors que nous ne voulons pas forcément qu'une telle similarité apparaisse. Pour cela l'article [3] "Linear spatial pyramid matching using sparse coding for image classification" préconise de créer plusieurs niveaux de représentations de BoW augmentant l'importance de l'information spatiale au fur et à mesure que l'information est codée dans les "étages" supérieurs de la pyramide. Il faut alors pondérer l'importance de chaque niveau de la pyramide en fonction de l'importance que nous voulons au niveau spatiale. 
\subsection{Différences dans les méthodes de coding/pooling pour la formation des BoW}

\subsection{Normalisation des BoW ($l_1$ vs  $l_2$)}
\section{Conclusion}

\end{document}